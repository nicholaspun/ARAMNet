\subsection{Solving the Two-Player Game}\label{Section:Two-Player}
Let's restrict ourselves to the case of two players: $P_1$ and $P_2$, with individual pools $\U_1$ and $\U_2$.
To get rid of some boring cases, for the remainder of this section, we will further assume that $\U_1 \cap \U_2 \neq \emptyset$ and $\U_1 \neq \U_2$.
It's easy to see that failure of the first assumption puts us in the case where selections are independent and failure of the second assumption puts us in the case where every owns the universal pool.

With those assumptions out of the way, what remains are two cases:
\begin{itemize}
    \item $\U_1 \subset \U_2$: $\U_1$ is contained in $\U_2$
    \item $\U_1 \setminus \U_2 \neq \emptyset,\ |\U_1| < |\U_2|$: Both pools contain elements unique to themselves. Note that if $|\U_1| = |\U_2|$, then the ordering doesn't matter. 
\end{itemize}

\begin{proposition}
    If $\U_1 \subset \U_2$ then $P_2$ should go first
\end{proposition}
\begin{proof}
    We show that when $P_2$ goes first, this induces a uniform distribution for both players.
    Let $\U_1 = \{a_1, \ldots a_m\}$ and $\U_2 = \U_1 \cup \{b_1, \ldots b_m\}$.
    
    Since $P_2$ goes first, $\Pr(P_2 \text{ selects } a_i) = \Pr(P_2 \text{ select } b_j) = \frac{1}{n + m}$ for any $i, j$

    Then, for $P_1$:
    \begin{align*}
        \Pr(P_1 \text{ selects } a_i) 
        &= \sum_j \Pr(P_1 \text{ selects } a_i \vert P_2 \text{ selects } b_j)\Pr(P_2 \text{ selects } b_j) \\
        &\quad + \Pr(P_1 \text{ selects } a_i \vert P_2 \text{ selects } a_i)\Pr(P_2 \text{ selects } a_j) \\
        &\quad + \sum_{k \neq i} \Pr(P_1 \text{ selects } a_i \vert P_2 \text{ selects } a_k)\Pr(P_2 \text{ selects } a_k) \\\
        &= \sum_j \frac{1}{n}\cdot\frac{1}{n + m} + 0 + \sum_{k \neq i}\frac{1}{n-1}\cdot\frac{1}{n + m} \\
        &= \frac{m}{n(n + m)} + \frac{n - 1}{(n-1)(n+m)} \\
        &= \frac{1}{n} 
    \end{align*}

    So, the selection probability is uniform for both players.
\end{proof}

\begin{proposition}\label{prop:two-player-second-case}
    If $\U_1 \setminus \U_2 \neq \emptyset$ and $|\U_1| < |\U_2|$, then $P_2$ should go first.
\end{proposition}
\begin{proof}
    See \Cref{Appendix:Proof-Of-Two-Player-Prop}. The proof follows the same idea as above, but the computation is a bit more involved.
\end{proof}

In either case, we see that $P_2$ should always go first.
Of course, since the assignment of the pools was arbitrary, the actual property that leads to this result is that $|\U_2 \setminus \U_1| > |\U_1 \setminus \U_2|$.
In words: after removing the elements within the intersection of $\U_1$ and $\U_2$, $\U_2$ has more elements left over.
Then, when $P_2$ goes first, they will have a higher probability of selecting an element \it{not} owned by $P_1$, making it such that the two player's choices are independent of each other.
As hinted in \Cref{Sec:Problem-Definition}, we will refer to this approach of ``picking the player with the most leftover elements to go first'' as the ``greedy'' approach.
We state the conjecture that will drive the remainder of the paper:
\begin{conjecture}
    The greedy approach yields an ordering that minimizes the Hellinger distance between every player's selection distribution to the uniform distribution.
\end{conjecture}