\subsection{Analyzing the Three-Player Game and Beyond}

The two-player case is fairly straightforward since we reduced it to two real cases.
However, as we move on to the more general case, we will find that the number of cases blow up exponentially.
For example, with 3 players, we may see the following complex intersections:

\begin{minipage}{\textwidth}
    \centering
    \begin{tikzpicture}
        \node[vertex, minimum size=2cm] (a) at (0,0) {};
        \node[vertex, minimum size=2cm] (b) at (1,0) {};
        \node[vertex, minimum size=2cm] (c) at (0.5,1) {};

        \node[vertex, minimum size=2cm] (a) at (4,0) {};
        \node[vertex, minimum size=1cm] (b) at (4,0) {};
        \node[vertex, minimum size=2cm] (c) at (5,0) {};

        \node[vertex, minimum size=2cm] (a) at (8,0) {};
        \node[vertex, minimum size=2cm] (b) at (8.8,0) {};
        \node[vertex, minimum size=2cm] (c) at (9.6,0) {};
    \end{tikzpicture}
\end{minipage}

As such, our approach of trying to break down the probabilities using law of total probability quickly becomes unwieldy.

There were attempts to model this as a sort-of stochastic process.
However, such a model would require memory of arbitrary length since we can always create arbitrarily far dependencies.
For example, we can consider a game where only two players have an element $x$ in their pools, but the ordering places them first and last.
This makes it seem difficult to model (most techniques only care about the previous state or a history of finite length), however, we admit here that the author also has limited knowledge in probability theory.

Nevertheless, we work with the intuition that we built when analyzing the two-player case.
In particular, we want to get a better handle on the idea that a greedy approach may yield the best ordering.
Intuitively speaking, by using a greedy approach, at each step we maximize the probability of making independent selections.
As such, it feels like this will bring us towards a solution most close to uniform.
In the next section, we will continue down this line of thought by trying to find empirical evidence of our conjecture.