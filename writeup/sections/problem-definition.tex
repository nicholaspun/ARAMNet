\section{Problem Definition}\label{Sec:Problem-Definition}

In this section, we will present a generalization of the problem described in \Cref{Section:Motivation} and also refine the questions we want to answer in this paper.

Let $\U = \{u_1, \ldots, u_n\}$ be the universal pool of $n$ elements and $\U_1, \ldots \U_m \subseteq \U$ be the individual pools for Players $P_1$ through $P_m$ respectively.
Let $\sigma: [m] \longrightarrow [m]$ be an ordering of the $m$ players.
(So, $P_{\sigma(1)}$ is the player that goes first, $P_{\sigma(2)}$ second and so forth)
Suppose the $m$ players pick from $\U$ using the following scheme:

\IncMargin{2em}
\begin{algorithm}[H]
    \SetKwInOut{Input}{Input}\SetKwInOut{Output}{Output}
    
    \BlankLine
    \Input{Universal pool $\U$, player pools $\U_1, \ldots, \U$, and player ordering $\sigma$}
    \Output{An assignment of $P_1, \ldots, P_m$ to unique elements $u_{P_1}, \ldots, u_{P_m}$}
    \BlankLine

    $\U_{\mathtt{curr}} \leftarrow \U$

    \For{$i \leftarrow 1$ \KwTo $m$}{
        Select an element $u_j$ (uniformly at random) from $\U_{P_{\sigma(i)}} - \U_{\mathtt{curr}}$ and assign it to $P_{\sigma(i)}$
        
        $\U_{\mathtt{curr}} \leftarrow \U_{\mathtt{curr}} - \{u_j\}$
    }
\end{algorithm}
\DecMargin{2em}

We've already seen that changing the order of the players has an impact on the individual probability distributions.
(We will also show this more carefully in \Cref{Section:Mathematical-Approach})
Since we're ultimately playing a game, it's natural to want an ordering that maximizes fairness.
That is, we want an ordering of players such that the probability for player $P_i$ to select element $u \in \U_i$ is \it{close to} uniform $\left(\frac{1}{|\U_i|}\right)$.
By \it{close to}, we mean that the following metric should be minimized:

\begin{definition}[Hellinger Distance\footnote{See \url{http://encyclopediaofmath.org/index.php?title=Hellinger_distance&oldid=47206}}]
    Let $P = (p_1, \ldots, p_k)$, $Q = (q_1, \ldots, q_k)$ be two discrete probability distributions, their \it{Hellinger distance} is defined as:
    \[
        H(P, Q) = \frac{1}{\sqrt{2}}\sqrt{\sum_{i = 1}^k (\sqrt{p_i} - \sqrt{q_i})^2}
    \]
\end{definition}
\begin{remark}
    $0 \leq H(P,Q) \leq 1$ 
\end{remark}

Unfortunately, the question doesn't seem to yield a simple answer and so, while it will drive the research below, we may not see it answered by the end of paper.
Instead, we fall back to a simpler question and end up comparing two candidate orderings: the random ordering and the ``greedy'' ordering.

In \Cref{Section:Mathematical-Approach}, we will solve the two-player case and see why a greedy approach is a candidate solution.
We will also see why the general case is difficult to handle, leading to the idea of developing empirical evidence, which will be explored in \Cref{Section:Empirical-Approach}.
Finally, we will leave some possible directions to further this research in \Cref{Section:Outlook}